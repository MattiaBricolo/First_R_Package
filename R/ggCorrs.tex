% Options for packages loaded elsewhere
\PassOptionsToPackage{unicode}{hyperref}
\PassOptionsToPackage{hyphens}{url}
%
\documentclass[
]{article}
\usepackage{lmodern}
\usepackage{amssymb,amsmath}
\usepackage{ifxetex,ifluatex}
\ifnum 0\ifxetex 1\fi\ifluatex 1\fi=0 % if pdftex
  \usepackage[T1]{fontenc}
  \usepackage[utf8]{inputenc}
  \usepackage{textcomp} % provide euro and other symbols
\else % if luatex or xetex
  \usepackage{unicode-math}
  \defaultfontfeatures{Scale=MatchLowercase}
  \defaultfontfeatures[\rmfamily]{Ligatures=TeX,Scale=1}
\fi
% Use upquote if available, for straight quotes in verbatim environments
\IfFileExists{upquote.sty}{\usepackage{upquote}}{}
\IfFileExists{microtype.sty}{% use microtype if available
  \usepackage[]{microtype}
  \UseMicrotypeSet[protrusion]{basicmath} % disable protrusion for tt fonts
}{}
\makeatletter
\@ifundefined{KOMAClassName}{% if non-KOMA class
  \IfFileExists{parskip.sty}{%
    \usepackage{parskip}
  }{% else
    \setlength{\parindent}{0pt}
    \setlength{\parskip}{6pt plus 2pt minus 1pt}}
}{% if KOMA class
  \KOMAoptions{parskip=half}}
\makeatother
\usepackage{xcolor}
\IfFileExists{xurl.sty}{\usepackage{xurl}}{} % add URL line breaks if available
\IfFileExists{bookmark.sty}{\usepackage{bookmark}}{\usepackage{hyperref}}
\hypersetup{
  pdftitle={ggCorrs.R},
  pdfauthor={matti},
  hidelinks,
  pdfcreator={LaTeX via pandoc}}
\urlstyle{same} % disable monospaced font for URLs
\usepackage[margin=1in]{geometry}
\usepackage{color}
\usepackage{fancyvrb}
\newcommand{\VerbBar}{|}
\newcommand{\VERB}{\Verb[commandchars=\\\{\}]}
\DefineVerbatimEnvironment{Highlighting}{Verbatim}{commandchars=\\\{\}}
% Add ',fontsize=\small' for more characters per line
\usepackage{framed}
\definecolor{shadecolor}{RGB}{248,248,248}
\newenvironment{Shaded}{\begin{snugshade}}{\end{snugshade}}
\newcommand{\AlertTok}[1]{\textcolor[rgb]{0.94,0.16,0.16}{#1}}
\newcommand{\AnnotationTok}[1]{\textcolor[rgb]{0.56,0.35,0.01}{\textbf{\textit{#1}}}}
\newcommand{\AttributeTok}[1]{\textcolor[rgb]{0.77,0.63,0.00}{#1}}
\newcommand{\BaseNTok}[1]{\textcolor[rgb]{0.00,0.00,0.81}{#1}}
\newcommand{\BuiltInTok}[1]{#1}
\newcommand{\CharTok}[1]{\textcolor[rgb]{0.31,0.60,0.02}{#1}}
\newcommand{\CommentTok}[1]{\textcolor[rgb]{0.56,0.35,0.01}{\textit{#1}}}
\newcommand{\CommentVarTok}[1]{\textcolor[rgb]{0.56,0.35,0.01}{\textbf{\textit{#1}}}}
\newcommand{\ConstantTok}[1]{\textcolor[rgb]{0.00,0.00,0.00}{#1}}
\newcommand{\ControlFlowTok}[1]{\textcolor[rgb]{0.13,0.29,0.53}{\textbf{#1}}}
\newcommand{\DataTypeTok}[1]{\textcolor[rgb]{0.13,0.29,0.53}{#1}}
\newcommand{\DecValTok}[1]{\textcolor[rgb]{0.00,0.00,0.81}{#1}}
\newcommand{\DocumentationTok}[1]{\textcolor[rgb]{0.56,0.35,0.01}{\textbf{\textit{#1}}}}
\newcommand{\ErrorTok}[1]{\textcolor[rgb]{0.64,0.00,0.00}{\textbf{#1}}}
\newcommand{\ExtensionTok}[1]{#1}
\newcommand{\FloatTok}[1]{\textcolor[rgb]{0.00,0.00,0.81}{#1}}
\newcommand{\FunctionTok}[1]{\textcolor[rgb]{0.00,0.00,0.00}{#1}}
\newcommand{\ImportTok}[1]{#1}
\newcommand{\InformationTok}[1]{\textcolor[rgb]{0.56,0.35,0.01}{\textbf{\textit{#1}}}}
\newcommand{\KeywordTok}[1]{\textcolor[rgb]{0.13,0.29,0.53}{\textbf{#1}}}
\newcommand{\NormalTok}[1]{#1}
\newcommand{\OperatorTok}[1]{\textcolor[rgb]{0.81,0.36,0.00}{\textbf{#1}}}
\newcommand{\OtherTok}[1]{\textcolor[rgb]{0.56,0.35,0.01}{#1}}
\newcommand{\PreprocessorTok}[1]{\textcolor[rgb]{0.56,0.35,0.01}{\textit{#1}}}
\newcommand{\RegionMarkerTok}[1]{#1}
\newcommand{\SpecialCharTok}[1]{\textcolor[rgb]{0.00,0.00,0.00}{#1}}
\newcommand{\SpecialStringTok}[1]{\textcolor[rgb]{0.31,0.60,0.02}{#1}}
\newcommand{\StringTok}[1]{\textcolor[rgb]{0.31,0.60,0.02}{#1}}
\newcommand{\VariableTok}[1]{\textcolor[rgb]{0.00,0.00,0.00}{#1}}
\newcommand{\VerbatimStringTok}[1]{\textcolor[rgb]{0.31,0.60,0.02}{#1}}
\newcommand{\WarningTok}[1]{\textcolor[rgb]{0.56,0.35,0.01}{\textbf{\textit{#1}}}}
\usepackage{graphicx,grffile}
\makeatletter
\def\maxwidth{\ifdim\Gin@nat@width>\linewidth\linewidth\else\Gin@nat@width\fi}
\def\maxheight{\ifdim\Gin@nat@height>\textheight\textheight\else\Gin@nat@height\fi}
\makeatother
% Scale images if necessary, so that they will not overflow the page
% margins by default, and it is still possible to overwrite the defaults
% using explicit options in \includegraphics[width, height, ...]{}
\setkeys{Gin}{width=\maxwidth,height=\maxheight,keepaspectratio}
% Set default figure placement to htbp
\makeatletter
\def\fps@figure{htbp}
\makeatother
\setlength{\emergencystretch}{3em} % prevent overfull lines
\providecommand{\tightlist}{%
  \setlength{\itemsep}{0pt}\setlength{\parskip}{0pt}}
\setcounter{secnumdepth}{-\maxdimen} % remove section numbering

\title{ggCorrs.R}
\author{matti}
\date{2020-11-04}

\begin{document}
\maketitle

@title ACF, PACF and CCF

@description It uses the \pkg{ggplot2} package to create three usefull
functions in time series analysis: ACF, PACF and CCF.

@param x,y Variables of interest. @param min\_lag,max\_lag Corresponding
to the minimum lag and the maximum lag. Default values are
\code{min_lag=-20} and \code{max_lag=20}. @param plot\_type Specify the
type of the plot. You can choose \code{"ACF"} (default) for the
autocorrelation function, \code{"PACF"} for the partial autocorrelation
function and \code{"CCF"} for the cross-correlation function. @param
print\_vals See below. The default is \code{FALSE}. @param alpha The
type-I error. The default value is \code{alpha=.05}.

@return Return the desired plot. If \code{print_vals} is \code{TRUE} it
returns the LAG sequence and the relative ACF (or PACF or CCF) values.

The confidence intervals are calculated using the asymptotic
distribution of the correlation estimator \eqn{\hat{\rho}~N(0,T^{-1})}.
Therefore \deqn{IC(alpha)=c(-1,1)*qnorm(1-alpha/2)\sqrt{T^{-1}}}.

@examples \# ::::::::::: \# \# TOY EXAMPLE \# \# ::::::::::: \#

library(``ggplot2'') set.seed(111) x \textless- rnorm(50) y \textless-
rchisq(50, 4) g \textless- ggCorrs(x) \# g \textless- ggCorrs(x,
print\_vals = T) ggCorrs(x, plot\_type = ``PACF'') ggCorrs(x, y,
plot\_type = ``CCF'')

@export ggCorrs

\begin{Shaded}
\begin{Highlighting}[]
\NormalTok{ggCorrs <-}\StringTok{ }\ControlFlowTok{function}\NormalTok{(x, y, }\DataTypeTok{min_lag =} \DecValTok{-20}\NormalTok{, }\DataTypeTok{max_lag =} \DecValTok{20}\NormalTok{, }\DataTypeTok{plot_type =} \StringTok{"ACF"}\NormalTok{, }\DataTypeTok{print_vals =}\NormalTok{ F, }\DataTypeTok{alpha =} \FloatTok{.05}\NormalTok{)\{}

  \CommentTok{# If ACF}
  \ControlFlowTok{if}\NormalTok{(plot_type }\OperatorTok{==}\StringTok{ "ACF"}\NormalTok{)\{}
\NormalTok{    N <-}\StringTok{ }\KeywordTok{NROW}\NormalTok{(x)}
\NormalTok{    acf.data =}\StringTok{ }\KeywordTok{acf}\NormalTok{(x, }\DataTypeTok{plot =}\NormalTok{ F)}
\NormalTok{    index =}\StringTok{ }\KeywordTok{which}\NormalTok{(acf.data}\OperatorTok{$}\NormalTok{lag[, }\DecValTok{1}\NormalTok{, }\DecValTok{1}\NormalTok{] }\OperatorTok\StringTok{ }\DecValTok{0}\OperatorTok{:}\NormalTok{max_lag)}
\NormalTok{    acf =}\StringTok{ }\KeywordTok{data.frame}\NormalTok{(}\DataTypeTok{lag =}\NormalTok{ acf.data}\OperatorTok{$}\NormalTok{lag[index, }\DecValTok{1}\NormalTok{, }\DecValTok{1}\NormalTok{],}
                     \DataTypeTok{acf.val =}\NormalTok{ acf.data}\OperatorTok{$}\NormalTok{acf[index, }\DecValTok{1}\NormalTok{, }\DecValTok{1}\NormalTok{])}
    \ControlFlowTok{if}\NormalTok{(print_vals }\OperatorTok{==}\StringTok{ }\NormalTok{T)\{}
\NormalTok{      Vals <-}\StringTok{ }\KeywordTok{list}\NormalTok{(acf}\OperatorTok{$}\NormalTok{lag, acf}\OperatorTok{$}\NormalTok{acf.val)}
      \KeywordTok{names}\NormalTok{(Vals) <-}\StringTok{ }\KeywordTok{c}\NormalTok{(}\StringTok{"LAG"}\NormalTok{, }\StringTok{"ACF"}\NormalTok{)}
      \KeywordTok{print}\NormalTok{(Vals)}
\NormalTok{    \}}
    \KeywordTok{return}\NormalTok{(}\KeywordTok{ggplot}\NormalTok{(acf,}
                  \KeywordTok{aes}\NormalTok{(}\DataTypeTok{x =}\NormalTok{ lag, }\DataTypeTok{y =}\NormalTok{ acf.val)) }\OperatorTok{+}
\StringTok{             }\KeywordTok{geom_bar}\NormalTok{(}\DataTypeTok{stat =} \StringTok{'identity'}\NormalTok{) }\OperatorTok{+}
\StringTok{             }\KeywordTok{labs}\NormalTok{(}\DataTypeTok{x =} \StringTok{"Lag"}\NormalTok{, }\DataTypeTok{y =} \StringTok{"ACF"}\NormalTok{)}\OperatorTok{+}
\StringTok{             }\KeywordTok{geom_hline}\NormalTok{(}\DataTypeTok{yintercept =} \KeywordTok{qnorm}\NormalTok{(}\DecValTok{1}\OperatorTok{-}\NormalTok{alpha}\OperatorTok{/}\DecValTok{2}\NormalTok{)}\OperatorTok{/}\KeywordTok{sqrt}\NormalTok{(N), }\DataTypeTok{color =} \StringTok{'blue'}\NormalTok{, }\DataTypeTok{linetype =} \StringTok{'dashed'}\NormalTok{) }\OperatorTok{+}
\StringTok{             }\KeywordTok{geom_hline}\NormalTok{(}\DataTypeTok{yintercept =} \OperatorTok{-}\KeywordTok{qnorm}\NormalTok{(}\DecValTok{1}\OperatorTok{-}\NormalTok{alpha}\OperatorTok{/}\DecValTok{2}\NormalTok{)}\OperatorTok{/}\NormalTok{(}\KeywordTok{sqrt}\NormalTok{(N)), }\DataTypeTok{color =} \StringTok{'blue'}\NormalTok{, }\DataTypeTok{linetype =} \StringTok{'dashed'}\NormalTok{))}
\NormalTok{  \}}

  \CommentTok{# If PACF}
  \ControlFlowTok{if}\NormalTok{(plot_type }\OperatorTok{==}\StringTok{ "PACF"}\NormalTok{)\{}
\NormalTok{    N <-}\StringTok{ }\KeywordTok{NROW}\NormalTok{(x)}
\NormalTok{    pacf.data =}\StringTok{ }\KeywordTok{pacf}\NormalTok{(x, }\DataTypeTok{plot =}\NormalTok{ F)}
\NormalTok{    index =}\StringTok{ }\KeywordTok{which}\NormalTok{(pacf.data}\OperatorTok{$}\NormalTok{lag[, }\DecValTok{1}\NormalTok{, }\DecValTok{1}\NormalTok{] }\OperatorTok\StringTok{ }\DecValTok{0}\OperatorTok{:}\NormalTok{max_lag)}
\NormalTok{    pacf =}\StringTok{ }\KeywordTok{data.frame}\NormalTok{(}\DataTypeTok{lag =}\NormalTok{ pacf.data}\OperatorTok{$}\NormalTok{lag[index, }\DecValTok{1}\NormalTok{, }\DecValTok{1}\NormalTok{],}
                      \DataTypeTok{pacf.val =}\NormalTok{ pacf.data}\OperatorTok{$}\NormalTok{acf[index, }\DecValTok{1}\NormalTok{, }\DecValTok{1}\NormalTok{])}
    \ControlFlowTok{if}\NormalTok{(print_vals }\OperatorTok{==}\StringTok{ }\NormalTok{T)\{}
\NormalTok{      Vals <-}\StringTok{ }\KeywordTok{list}\NormalTok{(pacf}\OperatorTok{$}\NormalTok{lag, pacf}\OperatorTok{$}\NormalTok{acf.val)}
      \KeywordTok{names}\NormalTok{(Vals) <-}\StringTok{ }\KeywordTok{c}\NormalTok{(}\StringTok{"LAG"}\NormalTok{, }\StringTok{"PACF"}\NormalTok{)}
      \KeywordTok{print}\NormalTok{(Vals)}
\NormalTok{    \}}
    \KeywordTok{return}\NormalTok{(}\KeywordTok{ggplot}\NormalTok{(pacf,}
                  \KeywordTok{aes}\NormalTok{(}\DataTypeTok{x =}\NormalTok{ lag, }\DataTypeTok{y =}\NormalTok{ pacf.val)) }\OperatorTok{+}
\StringTok{             }\KeywordTok{geom_bar}\NormalTok{(}\DataTypeTok{stat =} \StringTok{'identity'}\NormalTok{) }\OperatorTok{+}
\StringTok{             }\KeywordTok{labs}\NormalTok{(}\DataTypeTok{x =} \StringTok{"Lag"}\NormalTok{, }\DataTypeTok{y =} \StringTok{"PACF"}\NormalTok{)}\OperatorTok{+}
\StringTok{             }\KeywordTok{geom_hline}\NormalTok{(}\DataTypeTok{yintercept =} \KeywordTok{qnorm}\NormalTok{(}\DecValTok{1}\OperatorTok{-}\NormalTok{alpha}\OperatorTok{/}\DecValTok{2}\NormalTok{)}\OperatorTok{/}\KeywordTok{sqrt}\NormalTok{(N), }\DataTypeTok{color =} \StringTok{'blue'}\NormalTok{, }\DataTypeTok{linetype =} \StringTok{'dashed'}\NormalTok{) }\OperatorTok{+}
\StringTok{             }\KeywordTok{geom_hline}\NormalTok{(}\DataTypeTok{yintercept =} \OperatorTok{-}\KeywordTok{qnorm}\NormalTok{(}\DecValTok{1}\OperatorTok{-}\NormalTok{alpha}\OperatorTok{/}\DecValTok{2}\NormalTok{)}\OperatorTok{/}\NormalTok{(}\KeywordTok{sqrt}\NormalTok{(N)), }\DataTypeTok{color =} \StringTok{'blue'}\NormalTok{, }\DataTypeTok{linetype =} \StringTok{'dashed'}\NormalTok{))}
\NormalTok{  \}}

  \CommentTok{# If CCF}
  \ControlFlowTok{if}\NormalTok{(plot_type }\OperatorTok{==}\StringTok{ "CCF"}\NormalTok{)\{}
\NormalTok{    N <-}\StringTok{ }\KeywordTok{NROW}\NormalTok{(x)}
\NormalTok{    ccf.data =}\StringTok{ }\KeywordTok{ccf}\NormalTok{(x, y, }\DataTypeTok{plot =}\NormalTok{ F)}
\NormalTok{    index =}\StringTok{ }\KeywordTok{which}\NormalTok{(ccf.data}\OperatorTok{$}\NormalTok{lag[,}\DecValTok{1}\NormalTok{,}\DecValTok{1}\NormalTok{] }\OperatorTok\StringTok{ }\NormalTok{min_lag}\OperatorTok{:}\NormalTok{max_lag)}
\NormalTok{    ccf =}\StringTok{ }\KeywordTok{data.frame}\NormalTok{(}\DataTypeTok{lag =}\NormalTok{ ccf.data}\OperatorTok{$}\NormalTok{lag[index, }\DecValTok{1}\NormalTok{, }\DecValTok{1}\NormalTok{],}
                     \DataTypeTok{ccf.val =}\NormalTok{ ccf.data}\OperatorTok{$}\NormalTok{acf[index, }\DecValTok{1}\NormalTok{, }\DecValTok{1}\NormalTok{])}
    \ControlFlowTok{if}\NormalTok{(print_vals }\OperatorTok{==}\StringTok{ }\NormalTok{T)\{}
\NormalTok{      Vals <-}\StringTok{ }\KeywordTok{list}\NormalTok{(ccf}\OperatorTok{$}\NormalTok{lag, ccf}\OperatorTok{$}\NormalTok{acf.val)}
      \KeywordTok{names}\NormalTok{(Vals) <-}\StringTok{ }\KeywordTok{c}\NormalTok{(}\StringTok{"LAG"}\NormalTok{, }\StringTok{"CCF"}\NormalTok{)}
      \KeywordTok{print}\NormalTok{(Vals)}
\NormalTok{    \}}
    \KeywordTok{return}\NormalTok{(}\KeywordTok{ggplot}\NormalTok{(ccf,}
                  \KeywordTok{aes}\NormalTok{(}\DataTypeTok{x =}\NormalTok{ lag, }\DataTypeTok{y =}\NormalTok{ ccf.val)) }\OperatorTok{+}
\StringTok{             }\KeywordTok{geom_bar}\NormalTok{(}\DataTypeTok{stat =} \StringTok{'identity'}\NormalTok{) }\OperatorTok{+}
\StringTok{             }\KeywordTok{labs}\NormalTok{(}\DataTypeTok{x =} \StringTok{"Lag"}\NormalTok{, }\DataTypeTok{y =} \StringTok{"CCF"}\NormalTok{)}\OperatorTok{+}
\StringTok{             }\KeywordTok{geom_hline}\NormalTok{(}\DataTypeTok{yintercept =} \KeywordTok{qnorm}\NormalTok{(}\DecValTok{1}\OperatorTok{-}\NormalTok{alpha}\OperatorTok{/}\DecValTok{2}\NormalTok{)}\OperatorTok{/}\KeywordTok{sqrt}\NormalTok{(N), }\DataTypeTok{color =} \StringTok{'blue'}\NormalTok{, }\DataTypeTok{linetype =} \StringTok{'dashed'}\NormalTok{) }\OperatorTok{+}
\StringTok{             }\KeywordTok{geom_hline}\NormalTok{(}\DataTypeTok{yintercept =} \OperatorTok{-}\KeywordTok{qnorm}\NormalTok{(}\DecValTok{1}\OperatorTok{-}\NormalTok{alpha}\OperatorTok{/}\DecValTok{2}\NormalTok{)}\OperatorTok{/}\NormalTok{(}\KeywordTok{sqrt}\NormalTok{(N)), }\DataTypeTok{color =} \StringTok{'blue'}\NormalTok{, }\DataTypeTok{linetype =} \StringTok{'dashed'}\NormalTok{))}
\NormalTok{  \}}
\NormalTok{\}}
\end{Highlighting}
\end{Shaded}

\end{document}
